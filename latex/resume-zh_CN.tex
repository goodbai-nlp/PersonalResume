% !TEX TS-program = xelatex
% !TEX encoding = UTF-8 Unicode
% !Mode:: "TeX:UTF-8"

\documentclass{resume}
\usepackage{zh_CN-Adobefonts_external} % Simplified Chinese Support using external fonts (./fonts/zh_CN-Adobe/)
%\usepackage{zh_CN-Adobefonts_internal} % Simplified Chinese Support using system fonts
\usepackage{linespacing_fix} % disable extra space before next section
\usepackage{cite}

\begin{document}
\pagenumbering{gobble} % suppress displaying page number

\name{白雪峰}

\basicInfo{
  \email{bxf\_hit@163.com,xfbai@hit-mtlab.net} \textperiodcentered\ 
  \phone{(+86) 15754604524}}
 
\section{\faGraduationCap\  教育背景}
\datedsubsection{\textbf{哈尔滨工业大学}, 哈尔滨,黑龙江}{2017 -- 至今}
\textit{在读硕士}\ 计算机科学与技术, 预计 2019 年 7 月毕业
\datedsubsection{\textbf{哈尔滨工业大学}, 哈尔滨, 黑龙江}{2013 -- 2017}
\textit{学士}\ 计算机科学与技术

\section{\faCogs\ IT 技能}
% increase linespacing [parsep=0.5ex]
\begin{itemize}[parsep=0.5ex]
  \item 编程语言及平台: 比较熟悉Python编程;基本掌握C,C++,Matlab编程;常用平台: Linux,Windows
  \item 基本知识:
  熟悉HMM,CRF等经典模型,基本掌握词向量,CNN,AutoEncoder等深度学习模型
  \item 研究方向:
  对表示学习,双语词典抽取,跨语言文本分类任务有一定研究
\end{itemize}

\section{\faUsers\ 项目经历}
\datedsubsection{\textbf{改进基于GAN的无监督跨语言词向量生成模型} (进行中)}{2017年9月 -- 至今}
\role{Python,Pytorch}{实验室研究: }
\begin{onehalfspacing}
原模型通过引入对抗神经网络,无监督的训练两个生成器(转换矩阵)来对齐两个语言中高频词的词向量分布,最后将转换的结果分别作为两个语言在新的向量空间的词向量。我们通过使用自编码器代替原来的单层神经网络,并引入正则项Loss,期待得到更高质量的跨语言词向量

\end{onehalfspacing}

\datedsubsection{\textbf{基于KCCA的跨语言语义表示的研究与实现(本科毕设)}}{2017年1月 -- 2017年7月}
\role{Python \& Matlab, Linux}{个人项目,https://github.com/muyeby/GraduationProject}
通过复现和分析ACL2014,2016中相关论文的工作,发现了前人工作中的一些局限性(线性假设局限),
\begin{onehalfspacing}
从而提出了基于KCCA的非线性跨语言词向量生成模型,并在一些跨语言评测任务中相较于同类模型取得了不错的性能提升,相关成果已投稿ACM-TALLIP期刊
\end{onehalfspacing}

\datedsubsection{\textbf{scikit-learn机器学习库中文文档翻译项目}}{2017 年3月 -- 2017 年5月}
\role{Python}{开源项目,https://github.com/muyeby/scikit-learn-doc-cn}
\begin{onehalfspacing}
参与scikit-learn文档翻译项目,主要负责交叉分解相关章节以及一些公共的翻译工作。
\end{onehalfspacing}

\datedsubsection{\textbf{简单自然语言处理工具包(SFNLP)}}{2015 年9月 -- 2016 年7月}
\role{Python}{个人项目,https://github.com/muyeby/SFnlp}
\begin{onehalfspacing}
开源自然语言处理工具包,主要实现的是常用词法分析模型,具体涉及到中文分词、词性标注、命名实体识别、词性还原几个功能。用到的模型包括Perceptron,HMM,TnT,CRF等
\end{onehalfspacing}



% Reference Test
%\datedsubsection{\textbf{Paper Title\cite{zaharia2012resilient}}}{May. 2015}
%An xxx optimized for xxx\cite{verma2015large}
%\begin{itemize}
%  \item main contribution
%\end{itemize}



\section{\faHeartO\ 文章\&获奖}
\datedline{\textit{Improving Vector Space Word Representations Via Kernel Canonical Correlation Analysis}, ACM Transactions on Asian and Low-Resource Language Information Processing(TALLIP)}{第一轮review完成}
\datedline{\textit{哈尔滨工业大学校百优毕业论文}}{2017年6月}
\datedline{\textit{哈尔滨工业大学校年度科技创新一等奖}}{2014年3月}

\section{\faInfo\ 其他}
% increase linespacing [parsep=0.5ex]
\begin{itemize}[parsep=0.5ex]
  \item 技术博客: http://muyeby.github.io/
  \item GitHub: https://github.com/muyeby
\end{itemize}

%% Reference
%\newpage
%\bibliographystyle{IEEETran}
%\bibliography{mycite}
\end{document}
